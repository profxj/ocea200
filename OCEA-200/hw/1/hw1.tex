%%%%%%%%% MASTER -- compiles the 4 sections
\documentclass[11pt,letterpaper]{article}
\usepackage[compact]{titlesec} 
\usepackage{wrapfig}

%\usepackage[]{savetrees} 
%\pagestyle{myheadings}        
%\markright{Joseph F. Hennawi}
%\markboth{Joseph F. Hennawi}{Joseph F. Hennawi \hspace*{8cm}Research Plan\
% \ \ }


%\parindent=0.0in
%%%%%%%%%%%%%%%%%%%%%%%%%%%%%%%%%%%%%%%%%%%%%%%%%%%%%%%%%%%%%%%%%%%%%%%%%
%%%%%%%%%% EXACT 1in MARGINS %%%%%%%                                   %%
%\setlength{\hoffset}{0.13in}     %%                                   %%
%\setlength{\voffset}{0.2in}     %%                                   %%
%\setlength{\textwidth}{6.4in}     %%                                   %%
%\setlength{\oddsidemargin}{0in}   %% (It is recommended that you       %%
%\setlength{\evensidemargin}{0in}  %%  not change these parameters,     %%
%\setlength{\textheight}{8.95in}    %%  at the risk of having your       %%
%%\setlength{\textheight}{8.5in}    %%  at the risk of having your       %%
%%\setlength{\topmargin}{0in}   %%  proposal dismissed on the basis  %%
%\setlength{\topmargin}{-0.3in}   %%  proposal dismissed on the basis  %%
%\setlength{\headheight}{0in}      %%  of incorrect formatting!!!)      %%
%\setlength{\headsep}{0.5in}         %%                                   %%
%\setlength{\footskip}{0.5in}       %%                                   %%
%%%%%%%%%%%%%%%%%%%%%%%%%%%%%%%%%%%%                                   %%
%\newcommand{\required}[1]{\section*{\hfil #1\hfil}}                    %%
%\renewcommand{\refname}{\hfil References Cited\hfil}                   %%
%\bibliographystyle{plain}                                              %%
%%%%%%%%%%%%%%%%%%%%%%%%%%%%%%%%%%%%%%%%%%%%%%%%%%%%%%%%%%%%%%%%%%%%%%%%%

%\addtolength{\parskip}{1.7mm} 

%PUT YOUR MACROS HERE
%\def \kms            {~{\rm km~s}^{-1}}
%\def \hkpc      {h^{-1}{\rm\ kpc}}
%\def \kpc      {\rm\ kpc}
%\def \arcmin     { ^{\prime} }
%\def \arcsec    {^{\prime\prime}}
%\def\cm#1{\, {\rm cm^{#1}}}
%\newcommand{\mnhi}{N_{\rm HI}}
%\usepackage{setspace}

%\usepackage[normalmargins,normalleading,normalfloats]{savetrees} 
%\input defs.tex
%\linespread{0.9}

\usepackage{latexsym}
\usepackage{fancybox}
\usepackage{graphicx}
\usepackage{amssymb}
\usepackage{color}
%\usepackage{ulem}
\usepackage{float}
\usepackage{url}
 

\usepackage{amsmath}
\usepackage{epsfig,sidecap,natbib}
\newcommand{\apj}{ApJ}%                                         % Journal abbreviations
\newcommand{\apjs}{ApJS}
\newcommand{\apjl}{ApJL}
\newcommand{\aap}{A{\&}A}
\newcommand{\aaps}{A{\&}AS}
\newcommand{\mnras}{MNRAS}
\newcommand{\aj}{AJ}
\newcommand{\araa}{ARAA}
\newcommand{\pasp}{PASP}
\newcommand{\nat}{Nature}
\newcommand{\jcap}{JCAP}
\newcommand{\procspie}{SPIE}

\newcommand{\npair}{200+}
\newcommand{\nnir}{100}
\newcommand{\ntpe}{2000}
\newcommand{\nheii}{18}
%\newcommand{\mhmpc}{h^{-1} \, \rm Mpc}
%\newcommand{\hmpc}{$\mhmpc$}
\def \cgssb {{\rm\,erg\,s^{-1}\,cm^{-2}\,arcsec^{-2}}}
\def \mwlya {W_{\rm Ly\alpha}}
\newcommand{\mtheii}{\tau_{\rm HeII}^{\rm eff}}
\newcommand{\theii}{$\mtheii$}
\newcommand{\mrphys}{R_\perp}
\newcommand{\rphys}{$\mrphys$}
\def\ion#1#2{#1\,{\sc #2}}
\newcommand{\jjackpot}{SDSSJ0841+3921}


\pretolerance=10000
\textwidth=6.4in
\textheight=8.95in
\voffset = 0.in
%\voffset = -0.3in  % For my printer
\topmargin=0.0in
\headheight=0.00in
\hoffset = 0.0in
%\hoffset = 0.33in  %  For my printer
\headsep=0.00in
\oddsidemargin=0in
\evensidemargin=0in
\parindent=2em
\parskip=0.2ex
 
\renewcommand{\baselinestretch}{1.03}

\special{papersize=8.5in,11in}

%\addtolength{\parskip}{1.7mm} 

%PUT YOUR MACROS HERE
%\def \kms            {~{\rm km~s}^{-1}}
%\def \hMpc      {h^{-1}{\rm\ Mpc}}
\def \hkpc      {h^{-1}{\rm\ kpc}}
\def \kpc      {\rm\ kpc}
\def \arcmin     { ^{\prime} }
%\def \arcsec    {^{\prime\prime}}
%\def\cm#1{\, {\rm cm^{#1}}}
%\newcommand{\mnhi}{N_{\rm HI}}
%\usepackage{setspace}

%\usepackage[normalmargins,normalleading,normalfloats]{savetrees} 
\input ../../defs.tex
%\linespread{0.9}


\begin{document}

\title{Homework 1}

\maketitle

\begin{center}
J. Xavier Prochaska
\end{center}

\begin{enumerate}
\item Read -- Done

\item Ocean Basins

\begin{enumerate}

\item Table

\begin{table}[ht]
\caption{Ocean Basin Data}
\label{tab:dla}
\begin{center}
\begin{tabular}{cccc}
\hline
Ocean & Area (km$^2$) & Percentile \\
\hline
Pacific & 168,723,000 & 46.6 \\
Atlantic & 85,133,000 & 23.5 \\
Indian & 70,560,000 & 19.5 \\
Southern & 21,960,000 & 6.1 \\
Arctic & 15,558,000 & 4.3 \\
\hline
\end{tabular}
\end{center}
\end{table}
Taken from \url{https://en.wikipedia.org/wiki/Ocean#Oceanic_divisions}


\item Drawing -- Am taking a zero on this one.

\item Depth of shelfbreak -- "about 130m" from Talley Ch. 2;  \\
Top of a mid-ocean ridge -- $\approx 4500$m;  Figure 2.5b of Talley

\item The Ross Sea is in the Southern Ocean, south of the Pacific (Figure 2.8; Talley) \\
The Weddell Sea is also in the Southern Ocean, south of the Atlantic (Figure 2.9; Talley)
The Ninety-East ridge is in the Indian Ocean and runs roughly along the 90$^\circ$E meridian
(Figure 2.10; Talley)

\end{enumerate}

\item Scalars and Vectors

\begin{enumerate}
  \item Temperature
  \begin{enumerate}
    \item $T$ is a function of time $t$, i.e. $T(t)$
    \item Each $T$ measurement is a scalar.  
    \item $T$ is now a function of time and position.  It is a vector $\bar T(t, \bar r)$
  \end{enumerate}

  \item Heading $\theta = 120^\circ$ at $v = 5$\,knots.
  \begin{enumerate}
    \item $v_E = v \, \cos(90-\theta) = 8.02 \, \rm km/hr = 2.22 \, \rm m/s$
    \item $v_N = v \, \sin(90-\theta) = -4.63 \, \rm km/hr = -1.29 \, \rm m/s$
  \end{enumerate}

  \item 14 knots E, 9 knots S
  \begin{enumerate}
    \item $v = \sqrt{v_E^2 + v_S^2} = 8.67 \, \rm m/s$
    \item $\theta = \tan^{-1}(v_E/(-v_S)) = 122.73^\circ$
  \end{enumerate}

  \item Heading to Hawaii
  \begin{enumerate}
    \item $\theta = \tan^{-1}(\delta_E/\delta_N) = 123.7^\circ$
  \end{enumerate}

  \item Dot product
  \begin{enumerate}
    \item $\bar v_1 = (10., 0.) \, {\rm m/s}; \bar v_2 = (0., 50.) \, {\rm m/s}$.  \\
    The dot product is $\bar v_1 \cdot \bar v_2 = 10*0. + 0.*50. = 0 \, \rm m^2/s^2$
    \item The dot product will be maximal when $\bar v_2 \parallel \bar v_1$, i.e. rotate it $-90^\circ$
    in the standard $x,y$ convention.  Then $\bar v_1 \cdot \bar v_2 = 500 \, \rm m^2/s^2$
  \end{enumerate}

  \item Cross product
  \begin{enumerate}
    \item I presume we are now in an $x-y-z$ cartesian system.
    \item $\bar v_1 = (10., 0., 0.) \, {\rm m/s}; \; \bar v_2 = (0., 50., 0.) \, {\rm m/s}$.  \\
    The cross product is $\bar v_1 \times \bar v_2 = (10.*50. - 0.*0.) \, \hat z + 0. \, \hat y 
    + 0. \hat x = 500 \, {\rm m^2/s^2} \, \hat z$ 
    \item This is already maximal as the cross-product is maximum because $\bar v_1 \perp \bar v_2$
  \end{enumerate}

\end{enumerate}


\end{enumerate}

%\include{NSFsumm}
%\setcounter{page}{1}
%\setcounter{page}{1}
%\include{NSFrefs}
%\setcounter{page}{1}
%\include{NSFbio}
%\required{Project Description}


%\begin{center}
%  \begin{large}
%    {\bf Quasars Probing Quasars}
%  \end{large}
%\end{center}



\end{document}